\documentclass[11pt,letterpaper]{article}
\usepackage{multicol} 


% PAQUETES %%%%%%%%%%%%%%%%%%%%%%%%%%%%%%%%%%%%%%%%%%%%%%%%%%%%%%%%%%%%%%%%%%%%

\usepackage[utf8]{inputenc}
\usepackage{graphicx}
\usepackage{listings}
\usepackage{xcolor}
\usepackage{multicol}
\usepackage{anysize} %Permitir distintas medidas de margenes
\usepackage{framed}
\usepackage{fancyhdr}
\usepackage{float}


\fancyhf{}
\rhead{ING2104 - Fundamentos de Economía\\2022-10\vspace{0.1cm}}
\lhead{\includegraphics[scale=0.18]{ing.png}}
\lfoot{\scriptsize abanduc@miuandes.cl}
\rfoot{\thepage}

\renewcommand{\footrulewidth}{0.25pt}

\addtolength{\headheight}{1.4cm}


\newcommand{\evaluationtitle}[2]{
	% T\'itulo
	\begin{center}
	\vspace{1ex}\Large #1\\
	\vspace{1ex}\small #2
	\end{center}
}

\pagestyle{fancy}

\begin{document}

\begin{center}
    \textbf{\Large{Ayudantía 6}}\\\\
    Ayudantes: Antonia Banduc, Vicente Muñoz, Hernan Venegas, Roberto Witting
    
\end{center}

\begin{enumerate}
\item Imagine que su vecino está formando una banda de rock y compra una batería y una guitarra eléctrica, pero quienes tocan la guitarra y batería no son buenos, por lo que practican todos los días al lado.
\begin{enumerate}
    \item Mencione los costos privados y externos.
    \item Apoyándose en un gráfico, explique por qué es una externalidad.
    \item Analice y explique tres posibles soluciones.
\end{enumerate}


\item El mercado del papel en un país tiene las siguientes funciones oferta y demanda: $\mathrm{Q_d=162000-3000P}$ y  $\mathrm{Q_0=37000+2000P}$, donde $\mathrm{Q}$ es la cantidad de papel en lotes y $\mathrm{P}$ su precio. Hoy en día no se regula la contaminación que producen estas fábricas, causando estragos en la comunidad. El costo marginal externo asociado para este caso es $\mathrm{CMgE=0,001Q}$.
    \begin{enumerate}
        \item Calcular precio y cantidad de equilibrio sin regulación de parte del gobierno.
        \item Precio y cantidad óptima desde el punto de vista social.
        \item Calcule el costo social de la solución.
        \item Analice la alternativa de un impuesto y cuál sería su monto.
        \item Calcule los nuevos excedentes.
    \end{enumerate}

\item Suponga que existe un bien con las siguientes funciones de oferta y demanda $\mathrm{Q_d=6000-1000P}$ y $\mathrm{Q_o=2000P}$, respectivamente. Con las curvas indicadas anteriormente, resuelva:
    \begin{enumerate}
        \item Calcule el equilibrio de mercado.
        \item Ahora suponga que el gobierno decide intervenir porque cree que el precio de equilibrio es muy alto, y decide bajarlo a la mitad ¿Qué debe hacer?
        \item Si aplica un subsidio ¿De cuánto debe ser? ¿Se debe entregar a los productores o consumidores? 
    \end{enumerate}

\item Imagine una función de demanda agregada $\mathrm{P=100-2Q}$ y una función de oferta agregada $\mathrm{P=10+Q}$ Computa el par (p*,Q*) de equilibrio. El gobierno introdujo un impuesto de cuantía t sobre la función de demanda y calcule la nueva cantidad de equilibrio, el precio que recibe el productor y precio que paga el comprador $\mathrm{(Q_t;p_v;p_c)}$. Realice el mismo ejercicio introduciendo el mismo impuesto sobre la oferta y calcula el nuevo $\mathrm{(Q_t;p_v;p_c)}$. Demuestre que estas magnitudes son iguales en uno y otro caso.



\end{enumerate}


\end{document}
