\documentclass[11pt,letterpaper]{article}
\usepackage[utf8]{inputenc}
\usepackage[spanish]{babel}
\usepackage{amsmath}
\usepackage{amsfonts}
\usepackage{amssymb}
\usepackage[left=1in,right=1in,top=1in,bottom=1in]{geometry}
%Paquete para graficar
\usepackage{pgfplots}
\usepackage{multicol}
\usepackage{color}
\pgfplotsset{compat=1.8}
\usepackage{enumerate}

%Tiks
\usepackage{physics}
\usepackage{amsmath}
\usepackage{tikz}
\usepackage{mathdots}
\usepackage{yhmath}
\usepackage{cancel}
\usepackage{color}
\usepackage{siunitx}
\usepackage{array}
\usepackage{multirow}
\usepackage{amssymb}
\usepackage{gensymb}
\usepackage{tabularx}
\usepackage{extarrows}
\usepackage{booktabs}
\usetikzlibrary{fadings}
\usetikzlibrary{patterns}
\usetikzlibrary{shadows.blur}
\usetikzlibrary{shapes}



\input{Common.tex}

\pagestyle{fancy}

\begin{document}


\begin{center}
    \textbf{\Large{Ayudantía 12}}\\
    Ayudantes: Antonia Banduc, Vicente Muñoz, Hernan Venegas, Roberto Witting
\end{center}


\section*{Ejercicio 1}
Imagine que un amigo le pide prestado \$80.000 y le devuelve \$95.000 a final de año. Suponga que la tasa de inflación ese año fue de 10\%.
\begin{enumerate}
    \item Calcule la tasa de interés nominal que su amigo le pagaría.
\item Calcule la tasa de interés real que su amigo le pagaría.
\item Si su amigo se demora dos años en pagarle y mantiene la tasa de interés, ¿Cuánto recibiría usted?
\item  Usted le presta el dinero a su amigo en UF, cuando vale \$30.000, si su amigo mantiene la tasa de
interés real y el prestamo es por un año, ¿Cuánto dinero recibe?
\item Ahora suponga que su amigo le ofrece devolverle \$110.000 al finalizar el tercer año. ¿Conviene darle
el préstamo?
\end{enumerate}

\section*{Ejercicio 2}

Como premio de lotería ha recibido \$100.000, dado que valora más el dinero mañana, ha
decidido depositar su capital. El banco le ofrece las siguientes opciones por 4 años:
\begin{itemize}
    \item Depositar los \$100.000 con una tasa anual efectiva del 12\%.
    \item Colocar los \$100.000 en un cuenta con interés efectiva trimestral del 3\%. 
\end{itemize}

¿Qué opción le ocnviene más?

%Con la primera opción, luego de un año, tendría $$100.000 \cdot (1+0.12) = 112.000$$
%Pero con la segunda opción, luego de 1 año (4 trimestres) tendría
%$$100.000\cdot(1+0.03)^4 = 112.550$$
%Entonces, le conviene la segunda opción, ya que capitaliza el interés más períodos.


\section*{Ejercicio 3}
Suponga que hace un año hizo un depósito de \$500.000 en una cuenta, justo un año después usted tiene
\$750.000.
\begin{enumerate}
    \item ¿Cuál fue la tasa de interés nominal mensual que obtuvo?
    \item ¿Cuál fue la tasa de interés real que obtuvo, si la inflación fue de 5\%?
\end{enumerate}

\section*{Ejercicio 4}
Si hago un depósito de \$100.000 en una cuenta que entrega un interés de 10\% anual, ¿Cuánto dinero habrá en la cuenta después de 3 años?

\section*{Ejercicio 5}
Si tengo \$250.000 y quiero que se transformen en \$320.000 en un fondo que entrega 2.5\% de rentabilidad anual, ¿Cuánto tiempo debo esperar?


\end{document}