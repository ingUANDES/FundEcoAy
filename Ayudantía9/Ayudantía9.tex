\documentclass[11pt,letterpaper]{article}
\usepackage{multicol} 


% PAQUETES %%%%%%%%%%%%%%%%%%%%%%%%%%%%%%%%%%%%%%%%%%%%%%%%%%%%%%%%%%%%%%%%%%%%

\usepackage[utf8]{inputenc}
\usepackage{graphicx}
\usepackage{listings}
\usepackage{xcolor}
\usepackage{multicol}
\usepackage{anysize} %Permitir distintas medidas de margenes
\usepackage{framed}
\usepackage{fancyhdr}
\usepackage{float}


\fancyhf{}
\rhead{ING2104 - Fundamentos de Economía\\2022-10\vspace{0.1cm}}
\lhead{\includegraphics[scale=0.18]{ing.png}}
\lfoot{\scriptsize abanduc@miuandes.cl}
\rfoot{\thepage}

\renewcommand{\footrulewidth}{0.25pt}

\addtolength{\headheight}{1.4cm}


\newcommand{\evaluationtitle}[2]{
	% T\'itulo
	\begin{center}
	\vspace{1ex}\Large #1\\
	\vspace{1ex}\small #2
	\end{center}
}

\pagestyle{fancy}

\begin{document}

\begin{center}
    \textbf{\Large{ }}\\
    \textbf{\Large{Ayudantía 9}}\\
    \textbf{\Large{ }}\\
    Profesores: Sebastián Cea, Jouseline Salay y Jorge Arenas\\
    Ayudantes: Antonia Banduc, Vicente Muñoz, Roberto Witting y Hernán Venegas\\
    \textbf{\Large{ }}\\
    \textbf{\Large{ }}\\
\end{center}

\begin{enumerate}

\item Imagine que un mercado tiene las siguientes funciones de oferta y demanda laboral:

\begin{center}
$L^s$ = 10 + 10w y $L^d$ = 100 - 5w\\
\end{center}

\begin{enumerate}
\item Encuentre el nivel de equilibrio.\\

\item Suponga la fijacióon de un salario igual a 10. Determine que pasará con el equilibrio de mercado.\\
\textbf{\Large{ }}\\
\textbf{\Large{ }}\\
\end{enumerate}
 
 
\item En un país hay 1000 campos ganaderos competitivos, que producen carne y venden a \$2 el kg de carne. La función de producción de cada campo es:
    
\begin{center}
q = 100L - $L^2$\\
\end{center}  
    
donde q son los kg de carne producidos diaramente y L el número de trabajadores en un campo.   \\ 
    
\begin{enumerate}    
\item Calcule la demanda de trabajo de cada campo en función del salario (w).\\
\item Calcular demanda de trabajo del mercado.\\
\item Ahora considere que la oferta laboral del mercado está determinada por w = 25 + 0,003L. Encontrar el equilibrio del mercado de trabajo, cuantos trabajadores contrata cada campo y cuantos kg de carne producen.\\
\item Suponga ahora, que debido a la alta inmigración, la nueva función de oferta laboral de ese mercado es w = 10 + 0.003L. Encontrar el nuevo equilibrio del mercado laboral y explicar los efectos de la inmigración sobre el salario y empleo. Grafique.\\
\item Por último, considere que el gobierno fija un salario mínimo de \$95. Determine y grafique que sucede en ese mercado de trabajo. Ahora haga lo mismo pero con un salario mínimo de \$100.\\
\end{enumerate}
\textbf{\Large{ }}\\

\hfill
\item Los siguiente tabla contiene los ingresos (en \%) de un país agrupados por quintiles.
        
    \begin{table}[H]
    \centering                      % used for centering table
    \begin{tabular}{|c|c|}       
     \hline
    Quintil & Ingreso (\%) \\ \hline
     1  & 7,7  \\ \hline
     2  & 12,3 \\ \hline
     3  & 15,48 \\  \hline
     4  & 20,5 \\ \hline
     5  & 44,1 \\ \hline
    
    \end{tabular}
    \end{table}


\begin{enumerate}    
\item Grafique las líneas de equidad y desigualdad perfecta.\\
\item Graficar la curva de Lorenz.\\
\item Calcular coeficiente de Gini.\\

\end{enumerate}




\end{enumerate}


\end{document}
