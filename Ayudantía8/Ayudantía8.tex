\documentclass[addpoints,answers]{exam}
\usepackage[spanish]{babel}
\usepackage[utf8]{inputenc}
\usepackage{color}
\usepackage{amsmath,amssymb,latexsym,color,graphicx}
\usepackage{multicol}
\usepackage{hyperref}
%This block of commented code translates default words to Spanish
%-------------------------------------------------------------
\pointpoints{punto}{puntos}
\bonuspointpoints{punto extra}{puntos extra}

\totalformat{Pregunta \thequestion: \totalpoints{} puntos}

\chqword{Pregunta}
\chpgword{Página}
\chpword{Página}
\chpword{Puntos}
\chbpword{Puntos extra}
\chsword{Puntos obtenidos}
\chtword{Total}

\footer{}{Página \thepage\ de \numpages}{}

%\boxedpoints
%-------------------------------------------------------------
\renewcommand{\solutiontitle}{\noindent\textbf{Solución:}\par\noindent}
\SolutionEmphasis{\color{blue}}

%Tikz
\usepackage{physics}
\usepackage{amsmath}
\usepackage{tikz}
\usepackage{mathdots}
\usepackage{yhmath}
\usepackage{cancel}
\usepackage{color}
\usepackage{siunitx}
\usepackage{array}
\usepackage{multirow}
\usepackage{amssymb}
\usepackage{gensymb}
\usepackage{tabularx}
\usepackage{extarrows}
\usepackage{booktabs}
\usetikzlibrary{fadings}
\usetikzlibrary{patterns}
\usetikzlibrary{shadows.blur}
\usetikzlibrary{shapes}


\begin{document}
%This code creates the text before the first question
%-------------------------------------------------------------------


\begin{multicols}{2}

\vspace*{0.24cm}\\
	\includegraphics[scale=0.3]{FAC_INGYCIEN.jpeg}\\
	\begin{center}
		\textbf{Fundamentos de Economía}\\
		\textsc{Primer semestre 2022}\\
		Profesor: Sebastián Cea, Jouseline Salay y Jorge Arenas\\
		Ayudantes: Antonia Banduc, Vicente Muñoz, Roberto Witting y Hernán Venegas
	\end{center}

\end{multicols}

\begin{center}

\sc Ayudantía 8

\end{center}


%Here, the questions begin
\begin{questions}

%First question below
\question La Asociación de Farmacéuticos de Barrio le solicitan al gobierno que establezca un impuesto
a las grandes cadenas de Farmacia, ya que al ubicarse cerca de ellos le generan una externalidad
negativa que les reduce sus utilidades. Comente.


\\
\\
\question Sean las curvas de oferta y demanda de un bien X:
\[Q^{s} = 2000P - 2000 \hspace{0.5cm}  y \hspace{0.5cm} Q^{d} = 10000 - 2000P\] \hfill \break
Su producción genera una externalidad positiva tal que si se tuviera en cuenta la curva sería: 
\hspace{0.5cm} \[Q^{s} = 2000P\] \hfill \break
a) Represente gráficamente la situación descrita indicando cuál sería el precio
correspondiente a la asignación eficiente.
\\
b) Calcule a cuánto asciende la externalidad positiva por unidad producida.

\\

\question Un grupo de pescadores de la X región se encuentra muy contento durantes estos
últimos días. En los periódicos ha aparecido la noticia que el compuesto denominado
“f” provoca un aceleramiento en el crecimiento de la anchoveta, un pez característico
de la región. La buena nueva es que dicho compuesto es eliminado al mar por la
compañía salmonera “Buen Salmón S.A.” al momento de realizar sus procesos de
elaboración de salmón enlatado, lo cual ha traído como consecuencia un aumento del
número de anchovetas pescadas durante las últimas semanas. Se ha realizado un
estudio el cual ha estimado que el impacto positivo de la producción de salmones
enlatados sobre la pesca de anchovetas está dada por
 \[f(q) = \frac{bq^2}{4}\] donde “q” es la
cantidad de latas producidas por “Buen Salmón SA”. La función de costos de la firma
salmonera es \[C(q) = a + bq^2\] y la demanda de mercado es \[P(q) = a - cq\] Suponga competencia perfecta.
\\

a) ¿Cuánto produce y a qué precio la firma salmonera?
\\
b) El encargado de pesca del conglomerado de pescadores atribuye el aumento de la
cantidad extraída de anchovetas a su excelente gestión, y en base a esto solicita al
grupo un aumento de su sueldo. ¿Se merece el aumento el encargado? Argumente
claramente su respuesta
\\
c) ¿Cuál es el óptimo social de producción de la firma salmonera? Grafique y explique





\end{questions}
\end{document}