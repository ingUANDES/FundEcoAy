\documentclass{article}

% Language setting
% Replace `english' with e.g. `spanish' to change the document language
\usepackage[spanish]{babel}

% Set page size and margins
% Replace `letterpaper' with `a4paper' for UK/EU standard size
\usepackage[letterpaper,top=2cm,bottom=2cm,left=3cm,right=3cm,marginparwidth=1.75cm]{geometry}

% Useful packages
\usepackage{amsmath}
\usepackage{graphicx}
\usepackage[colorlinks=true, allcolors=blue]{hyperref}

\title{Ayudantía 7}
\author{Ayudantes: Antonia Banduc, Vicente Muñoz,
Roberto Witting y Hernán Venegas
}

\begin{document}
\date{22 de Marzo 2022}
\maketitle

\begin{enumerate}
\item Preguntas de concepto:

a) Ubique en un gráfico de oferta y demanda el peso muerto producido por un impuesto.

b) Explique, con apoyo de gráficos, cómo afecta la elasticidad de la oferta y de la demanda en el peso muerto producido por un impuesto.

c) Supongamos que el mercado de botellas de vino tiene las siguientes funciones de oferta y
demanda:

$$Q_o =-200+2P \quad y\quad Q_d=1400-\frac{2}{3}P$$

La cantidad y precio de equilibrio de mercado son $P^*=\$600$ y  $Q^*=1000$. Calcule el peso muerto producido por un impuesto de $400\frac{\$}{botella}$.

\item Suponga el mercado de la carne de un país determinado por las siguientes funciones de oferta y demanda:

$$Q_o =-110+2P \quad y\quad Q_d=400-P$$

a) Calculé el precio y la cantidad óptima de mercado.

b) Ahora imagine que ese gobierno luego de varios estudios cree que es conveniente aplicar un subsidio de
\$210, grafique y calcule que ocurriría en ese mercado.

\item Imagine que el mercado del litio está dado por las siguientes funciones de oferta y demanda:
$$Q_o =4000+2000P \quad y\quad Q_d=160000-2000P$$

Pero ahora, para que sea posible la producción de ese litio, existe un costo marginal externo asociado a
su producción $CMgE = 0.2Q$.

a) Calcule precio y cantidad de equilibrio sin regulación.
b) Calcule precio y cantidad de equilibrio óptimos desde el punto de vista social.
c) Calcule el costo social de la solución en la parte a).
d) Estudie la opción de un impuesto para llegar al óptimo de la parte b).
e) Aplicado el nuevo impuesto, calcule los nuevos excedentes.

\item El gobierno de un pa´ıs quiere subsidiar con \$136 la compra de libros de economía, y para eso decide
aplicar un impuesto en el mercado del alcohol. Las funciones de demanda del mercado de los libros son:

$$Q_o =2P \quad y\quad Q_d=2400-6P$$

Y las funciones del mercado del alcohol son:

$$Q_o =P \quad y\quad Q_d=2000-P$$

¿Cuál debe ser el monto del impuesto a aplicar sobre los alcoholes?




\end{enumerate}

\end{document}

