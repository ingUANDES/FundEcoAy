\documentclass[11pt,letterpaper]{article}
\usepackage[utf8]{inputenc}
\usepackage[spanish]{babel}
\usepackage{amsmath}
\usepackage{amsfonts}
\usepackage{amssymb}
\usepackage[left=1in,right=1in,top=1in,bottom=1in]{geometry}
%Paquete para graficar
\usepackage{pgfplots}
\usepackage{multicol}
\usepackage{color}
\pgfplotsset{compat=1.8}
\usepackage{enumerate}

%Tiks
\usepackage{physics}
\usepackage{amsmath}
\usepackage{tikz}
\usepackage{mathdots}
\usepackage{yhmath}
\usepackage{cancel}
\usepackage{color}
\usepackage{siunitx}
\usepackage{array}
\usepackage{multirow}
\usepackage{amssymb}
\usepackage{gensymb}
\usepackage{tabularx}
\usepackage{extarrows}
\usepackage{booktabs}
\usetikzlibrary{fadings}
\usetikzlibrary{patterns}
\usetikzlibrary{shadows.blur}
\usetikzlibrary{shapes}



\input{Common.tex}

\pagestyle{fancy}

\begin{document}


\begin{center}
    \textbf{\Large{Ayudantía 10}}\\
    Ayudantes: Antonia Banduc, Vicente Muñoz, Hernan Venegas, Roberto Witting
\end{center}



\section*{Ejercicio 1}
\noindent \textbf{Impuesto al diesel}. El mercado local de diesel está representado por:

$$Q = \frac{a-P}{12.5}  \hspace{1cm} \text{ y}   \hspace{1cm}   Q = \frac{P-b}{11}$$

\noindent donde $P$ se encuentra en pesos chilenos (\$) por litro de combustible y $Q$ muestra los litros de combustible (en millones) que se transan en este mercado. Actualmente se cobra un impuesto de \$130 por litro de diesel.
\begin{enumerate}[a)]

\item Se le pide calcular la cantidad Q que se transa después de aplicado el impuesto. Exprese su resultado con dos decimales.


\item  Muestre detalladamente los cálculos del ejercicio anterior. ¿Hay diferencia entre el precio que observan oferentes y demandantes? Si es así, cuál es el precio que observa el productor y cuál es el precio que observa el consumidor? ¿A cuánto asciende la recaudación del gobierno? Si hay pérdida de eficiencia, ¿a cuánto asciende? Grafique.


\item En su cálculo, ¿aplicó el impuesto sobre el consumidor o sobre el productor? ¿Sería diferente el resultado si lo aplica al otro lado del mercado? ¿Por qué? Calcule la incidencia del impuesto y relaciónelo con la elasticidad precio propio de las curvas de oferta y demanda.


\item En la actualidad, hay argumentos a favor de aumentar el impuesto sobre el diésel y otros argumentos a favor de disminuir (al menos temporalmente) ese mismo impuesto. Mencione al menos un argumento que apoye el alza de este impuesto y uno que apoye la disminución del impuesto. Y finalmente, concluya a favor de qué argumento se encuentra usted.

\end{enumerate}



\section*{Ejercicio 2}
\noindent El mercado laboral de profesionales con capacidades tecnológicas se representa por la siguiente estructura de equilibrio entre oferta y demanda por precio P, de hora de trabajo (H):


$$P_d = 140 - \frac{H_d}{5}  \hspace{1cm} \text{ y}   \hspace{1cm}   P_o = \frac{H_o}{5}+80$$

\noindent El Ministerio del Trabajo evalúa subsidiar capacitaciones tecnológicas para fomentar la oferta de trabajadores en este mercado. Se estima que el valor del subsidio a la oferta por hora de trabajo es igual a la mitad del componente no variable de la función inversa.
\begin{enumerate}[a)]

\item Calcule la cantidad de horas de equilibrio en el mercado con el subsidio.


\item  Entregue los cálculos de equilibrio del ejercicio anterior, calcule y grafique los excedentes del consumidor y productor.


\item Calcule y grafique el costo en el que tiene que incurrir el estado para entregar este subsidio.


\end{enumerate}
\end{document}