\documentclass[11pt,letterpaper]{article}
\usepackage[utf8]{inputenc}
\usepackage[spanish]{babel}
\usepackage{amsmath}
\usepackage{amsfonts}
\usepackage{amssymb}
\usepackage[left=1in,right=1in,top=1in,bottom=1in]{geometry}
%Paquete para graficar
\usepackage{pgfplots}
\usepackage{multicol}
\usepackage{color}
\pgfplotsset{compat=1.8}
\usepackage{enumerate}

%Tiks
\usepackage{physics}
\usepackage{amsmath}
\usepackage{tikz}
\usepackage{mathdots}
\usepackage{yhmath}
\usepackage{cancel}
\usepackage{color}
\usepackage{siunitx}
\usepackage{array}
\usepackage{multirow}
\usepackage{amssymb}
\usepackage{gensymb}
\usepackage{tabularx}
\usepackage{extarrows}
\usepackage{booktabs}
\usetikzlibrary{fadings}
\usetikzlibrary{patterns}
\usetikzlibrary{shadows.blur}
\usetikzlibrary{shapes}



\input{Common.tex}

\pagestyle{fancy}

\begin{document}


\begin{center}
    \textbf{\Large{Ayudantía 11}}\\
    Ayudantes: Antonia Banduc, Vicente Muñoz, Hernan Venegas, Roberto Witting
\end{center}

\section*{Términos clave}

\textbf{Valor nominal: } Cálculo del valor de una inversión, activo servicio o producto utilizando el precio actual.\\

\textbf{Valor real: } Cálculo del valor de una inversión, activo servicio o producto utilizando como precio, un precio base del pasado.\\

\textbf{Tasa de interés: } Es un porcentaje de la operación que se realiza. Es un porcentaje que se traduce en un monto de dinero, mediante el cual se paga por el uso del dinero. \url{cmfchile.cl}\\

\textbf{PIB(Producto Interno Bruto): }Valor de mercado de todos los bienes y servicios finales producidos dentro de un país en un período determinado.\\
PIB por el lado del gasto:\\
$$Y = C + I + G + Xn$$
$Y:$ PIB\\
$C:$ Consumo\\
$I:$ Inversión\\
$G:$ Gasto Público\\
$Xn:$ Exportaciones netas\\

\textbf{PIB nominal:} Producción de bienes y servicios valuados a precios actuales.\\ 

\textbf{PIB real:} Producción de bienes y servicios valuados en un precio base. Con esto se busca eliminar el efecto de la inflación en el cálculo del PIB.\\

\textbf{Deflector del PIB: } medida del nivel de precios calculada como la raz ́on del PIB
nominal sobre el PIB real.\\

$$Deflactor_t = \frac{PIB_{nominal,t}}{PIB_{real,t}}$$

\textbf{Inflación: } Aumento generalizado y sostenido de los precios de los bienes y servicios existentes en el mercado durante un determinado período de tiempo. Comunmente, se comparan los precios de un perido base con los posteriores.

\newpage

\section*{Ejercicio 1}
\noindent Preguntas de concepto:

\begin{enumerate}[a)]
    \item Si los precios no cambian durante un periodo dado, el PIB nominal y el PIB real deberían coincidir. ¿Verdadero o falso?
    
    \item Un aumento en la población total va a aumentar el PIB per cápita de un país. ¿Verdadero o falso?
    
    \item El PIB es una medida perfecta del bienestar de las personas. ¿Verdadero o falso?
    
\end{enumerate}


\section*{Ejercicio 2}
\noindent Imagine que un amigo le pide prestado $80.000 y le devuelve $95.000 a final de año. Suponga que la
inflación ese año fue de 10 \%.

\begin{enumerate}[a)]

\item Calcule la tasa de interés nominal que su amigo le pagaría.

\item Calcule la tasa de interés real que su amigo le pagaría.

\item  Si su amigo se demora dos años en pagarle y mantiene la tasa de interés, ¿Cuánto recibiría usted?

\end{enumerate}



\section*{Ejercicio 3}
\noindent En una economía, sólo se producen autos y computadores. En la cuál se produjo lo siguiente en los años 2017,2018 y 2019:\\

\begin{table}[H]
\begin{tabular}{|l|l|l|l|l|}
\hline
Año & Precio Computadores & \begin{tabular}[c]{@{}l@{}}Cantidad Producida\\ Computadores\end{tabular} & Precio autos & \begin{tabular}[c]{@{}l@{}}Cantidad Producida\\ Autos\end{tabular} \\ \hline
2017 & 100 & 100 & 200 & 50 \\ \hline
2018 & 200 & 150 & 300 & 100 \\ \hline
2019 & 300 & 200 & 400 & 150 \\ \hline
\end{tabular}
\end{table}


\begin{enumerate}[a)]

\item Calcular el PIB nominal para los años 2017, 2018, 2019.

\item  Calcular el PIB real para los mismos años utilizando como año base el 2017.

\item Calcular el deflactor del PIB para los tres años mencionados anteriormente.

\item Calcular la tasa de inflación del mercado, utilizando el año 2017 como base. 

\end{enumerate}

\end{document}