\documentclass{article}

% Language setting
% Replace `english' with e.g. `spanish' to change the document language
\usepackage[spanish]{babel}

% Set page size and margins
% Replace `letterpaper' with `a4paper' for UK/EU standard size
\usepackage[letterpaper,top=2cm,bottom=2cm,left=3cm,right=3cm,marginparwidth=1.75cm]{geometry}

% Useful packages
\usepackage{amsmath}
\usepackage{graphicx}
\usepackage[colorlinks=true, allcolors=blue]{hyperref}

\title{Ayudantía 2}
\author{Ayudantes: Antonia Banduc, Vicente Muñoz,
Roberto Witting y Hernán Venegas
}

\begin{document}
\date{22 de Marzo 2022}
\maketitle

\begin{enumerate}
    \item Suponga que la siguiente tabla resume la economía de un país que produce únicamente dos bienes: 
    
    \begin{table}[h!]
    \centering
    \begin{tabular}{|l|l|l|l|l|l|}
\hline
                      & \textbf{A} & \textbf{B} & \textbf{C} & \textbf{D} & \textbf{E} \\ \hline
\textbf{Computadores} & 0          & 1          & 2          & 3          & 4          \\ \hline
\textbf{Autos}        & 10         & 8          & 6          & 3          & 0          \\ \hline
\end{tabular}
\end{table}

a) Dibujar la FPP.

b) Si esta economía se encuentra produciendo 1 computador, ¿cuál es el costo de oportunidad de comenzar a producir 4 computadores?

c) Calcular el costo de oportunidad del punto C al punto D.

d) ¿El rendimiento es creciente, decreciente o constante?

e) ¿Cómo se vería reflejado en la FPP un mejor rendimiento? (mayor producción)

    \item Una economía produce solo dos bienes, cobre y vino. Hay solo dos trabajadores y cada uno puede trabajar 10 horas diarias, en esas horas pueden producir 4 libras de cobre, o 6 litros de vino.

a) Dibuje la FPP individual y agregada.

b) Si se producen 50 litros de vino y 30 libras de cobre, ¿es eficiente? Si se producen 15 más de cada uno, ¿sería eficiente o alcanzable?

c) Ahora suponga que los trabajadores tienen capacidades diferentes, el primero puede producir 3 libras de cobre o 7 litros de vino, y el otro puede producir 5 libras de cobre o 4 litros de vino.
Plantee las FPP individuales y luego la agregada.

d) Si de la ecuación $x^2+y^2 = 100$ se obtiene la FPP. Explique por qué este ejemplo sería una generalización del caso anterior.

    \item Existe una economía que puede prestar servicios de limpieza y producir corbatas, la economía
en total dispone de 1000 horas de trabajo. El servicio de limpieza les toma ½ hora y producir la
corbata 5 horas.\\
a) ¿Cuántos servicios de limpieza se pueden ofrecer si destinan todos los recursos y la fuerza
laboral sólo a eso?\\
b) ¿Cuántas corbatas se pueden producir si se destinan todos los recursos y la fuerza laboral sólo a
eso?\\
c) Dibuje la FPP.\\
d) ¿Qué significa la pendiente de la FPP?\\
e) Calcule el costo de oportunidad de pasar de 100 a 200 servicios de limpieza.\\
    \item Suponga que hay dos países, el país A y el B. Al país A le toma 2 horas laborales producir 1 kg de
alimentos, y 10 horas laborales producir un computador, mientras que al país B le toma 10 horas
producir los mismos alimentos y 12 horas producir el mismo computador.\\
a) ¿Qué le diría a alguien que le dice “los habitantes del país A no tienen por que comerciar con los
del país B, hacen lo mismo mucho más rápido”?\\
b) ¿Cuál es el costo de oportunidad de producir 1 kg de alimento para el país A y B? ¿Y de
vestuario?\\
c) Dibuje las FPP de los países y comente.

    \item Sean Qo y Qd las ecuaciones de oferta y demanda de un bien X, respectivamente.
    $$Q_o = 50P-300$$
    $$Q_d=150-10P$$
    Calcule el precio y cantidad de equilibrio.
    
    \item Carlos es un entrenador de fútbol que puede entrenar y producir sólo dos posiciones, defensas (D) y mediocampistas (M). Carlos puede producir 15 defensas por mes o 5 mediocampistas por mes.

a) Escribe la ecuación que describe la producción de Carlos. (Asuma relación lineal)

b) Suponga que Carlos no está produciendo mediocampistas este mes. ¿Cuál es el costo de oportunidad de aumentar su producción de mediocampistas de 0 a 2?

c) Suponga que Carlos pensaba que iba a necesitar 3 defensas en el próximo mes peroresulta que finalmente necesitaba solo mediocampistas. ¿Cuál es el beneficio en mediocampistas que obtendrá ?

d) ¿Cuál es el costo de oportunidad de producir cada defensa?

e) ¿Cuál es el costo de oportunidad de producir cada mediocampista?

    \item  En una economía Pablo y Felipe ambos se dedican a la producción de bienes y lo hacen por separado. Ellos cosechan papas y producen carne de ganado. Sus producciones mensuales están dadas por la siguiente tabla:
    \begin{table}[h!]
\begin{tabular}{|l|l|l|}
\hline
 & \textbf{Producción máxima posible de papas} & \textbf{Producción máxima posible de carne} \\ \hline
\textbf{Pablo} & 30 kg & 10 kg \\ \hline
\textbf{Felipe} & 10 kg & 20 kg \\ \hline
\end{tabular}
\end{table}

a) ¿Quién tiene ventajas comparativas de producción?¿En qué? Demuestre con costos de oportunidad.

b)Grafique ambas FPP.

Asuma que ahora Pablo decide especializarse en la producción de papas,

c) ¿Cuántas papas puede producir ahora de forma mensual?

d) Debido a que ahora Pablo no tiene carne para comer, decide proponerle a Felipe comenzar un comercio de intercambio de productos. Proponga un escenario donde ambos salen beneficiados mejor que en su estado inicial.




    \item Usted gana 100.000 pesos en una rifa. Tiene dos opciones, invertir el dinero en un activo
que le proporcionará un interés de 5\% en un año más o gastar el dinero hoy en unas zapatillas exclusivas que no tendrán producción nuevamente. ¿ Considerando solo los valores monetarios, cuál es el costo de oportunidad de comprar las zapatillas? Si comprar las zapatillas significan para usted un beneficio sentimental llevado a valor monetario de 50.000, cuál es el costo de oportunidad de invertir el dinero?




\end{enumerate}

\end{document}